\documentclass[12pt,a4paper]{article}
\usepackage[utf8]{inputenc}
\usepackage[english]{babel}
\usepackage{csquotes}
\usepackage{setspace}
\usepackage{graphicx}
\usepackage{hyperref}
\usepackage[nolist]{acronym}
\usepackage[T1]{fontenc}
\usepackage{subfig}
\usepackage{placeins}

\onehalfspacing
\setlength{\parindent}{0pt}
\setlength{\parskip}{10pt}


\title{Project LIT - Outline}
\author{Erik Imgrund, Niklas Loeser, Andre Trump}
\date{\today}

\begin{document}
\maketitle

\section{Task Description}
What is the aim - in view of the specific phenomenon under consideration?
Identify linguistic biases of current models for NLI. Improve LMs by removing biases from the training process.

Hypotheses
Current language models are biased for NLI in the zero-shot and fine-tuned settings.
Removing biases from the dataset improves results in a less biased model.
A less biased model results in worse accuracy.

\section{Method}
How will we approach this aim?
First as a baseline we will test 

Which methods will/did you apply?

How to probe or fine-tune for your task?

\section{Models and Data Sets}
Select suitable data and resources
Additional resources to improve learning?
Use or create specific tests for targeted
evaluation (e.g., foiling, masking, ...)
Dataset statistics: classes, distributions, ...

\section{Experiments}

\subsection{Evaluation Metrics}

\subsection{Specific for Probing and Fine-Tuning}
\paragraph{Models to use}
\paragraph{Experiment Configuration}

\subsection{Experimental Settings}
\paragraph{Data Variations, Experiment Variation}
\paragraph{Baselines and Model Variants}

\section{Analysis} 
\subsection{Confusion Analysis}
\subsection{Visualizations}
\subsection{Interpretation Methods}

\end{document}