\documentclass[11pt]{article}

\usepackage{ACL2023}

\usepackage{times}
\usepackage{latexsym}
\usepackage[T1]{fontenc}
\usepackage[utf8]{inputenc}
\usepackage{microtype}
\usepackage{inconsolata}

\usepackage{listings}
\usepackage[nolist]{acronym}
\usepackage{placeins}
\usepackage{xcolor}
\usepackage{amsmath}
\usepackage{amssymb}
\usepackage{amsfonts}
\usepackage{etoolbox}
\usepackage{csquotes}
\usepackage{setspace}
\usepackage{graphicx}
\usepackage{booktabs}
\usepackage{multirow}
\usepackage{tablefootnote}
\usepackage{subcaption}


\makeatletter
\def\endthebibliography{%
  \def\@noitemerr{\@latex@warning{Empty `thebibliography' environment}}%
  \endlist
}
\makeatother

\widowpenalty=10000
\clubpenalty=10000
\parfillskip 0pt plus 0.75\textwidth
\makeatletter
\patchcmd{\@sect}{\begingroup}{\begingroup\parfillskip=0pt plus 1fil\relax}{}{}
\patchcmd{\@ssect}{\begingroup}{\begingroup\parfillskip=0pt plus 1fil\relax}{}{}
\makeatother


\title{Project LIT -- Debiasing \acs{NLI} models}

\author{André Trump \And
  Erik Imgrund \And
  Niklas Loeser }

\input{auxiliary/json.tex}

\begin{document}

\begin{acronym}
    \acro{CCG}{combinatory categorial grammar}
    \acro{e-SNLI}{Natural Language Inference with Natural Language Explanations}
    \acro{LIME}{Local Interpretable Model-Agnostic Explanations}
    \acro{LM}{Language Model}
    \acro{MCC}{Matthews correlation coefficient}
    \acro{MLM}{Masked Language Model}
    \acro{MultiNLI}{Multi-Genre Natural Language Inference}
    \acro{NLI}{Natural Language Inference}
    \acro{PLM}{Pretrained Language Model}
    \acro{RoBERTa}{Robustly Optimized {BERT} Pretraining Approach}
    \acro{SICK}{Sentences Involving Compositional Knowldedge}
    \acro{SNLI}{Stanford Natural Language Inference}
\end{acronym}

\maketitle
\begin{abstract}
Natural Language Inference (NLI) is the task of deciding on the entailment relation between a hypothesis and a premise. This task is very important for natural language understanding, as it can be used as a building block in more complex tasks. 

We find that the source of predictive performance for \acs{NLI} is found in the fine-tuning procedure. By analyzing the performance and plausibility of the predictions of a fine-tuned model, we conclude that the model is biased for specific phenomena. Furthermore, by evaluating a hypothesis-only model on the training dataset, we show that the training data contains biases that have different strengths when split across different linguistic phenomena.

We try multiple methods for mitigating data bias during training, succeeding in improving predictive performance and reducing bias in the explanations of the model. Furthermore, we recast data from the domain of question answering to the task of \acs{NLI} and apply linguistic techniques to find and generate new examples containing quantifier entailment. By training on the additional training data, the predictive performance is further increased on an easy evaluation set, but reduced on a hard evaluation set, indicating increased bias. We conclude, by confirming most of our hypotheses and showing our learnings.
\end{abstract}

\section{Motivation}

\ac{NLI} is the task of deciding the truthiness of a hypothesis, given a premise. If the hypothesis is true, it can be said to be \textbf{entailed} by the premise. If it is false, it is \textbf{contradictory} to the premise. Else the truthiness of the hypothesis cannot be determined. For each case, respectively one of the three classes \texttt{entailment}, \texttt{contradiction} or \texttt{neutral} is chosen.

% What is our aim?
Our aim is to identify biases in \acp{LM} for \ac{NLI} and mitigate those biases by removing biased data in the training procedure.

% What are our hypotheses?
We pose the following hypotheses:\vspace{-1.5em}
\begin{description}
  \item[H1] The fine-tuning procedure of \acp{LM} introduces biases for the \ac{NLI} task.\vspace{-0.7em}
  \item[H2] Mitigating biases from the datasets results in less biased \acp{LM}.\vspace{-0.7em}
  \item[H3] \acp{LM} with less bias make worse predictions for biased data. \vspace{-0.7em}% Will not name accuracy here, as accuracy is bad. We can name F1 and other score later.
  \item[H4] \acp{LM} with less bias make better predictions for data without the same bias.
\end{description}

w\cite{dissent}
\section{Method} \label{sec:method}
\subsection{Probing for \acs{NLI}} \label{sec:meth:probing}
To test the inherent classification performance of \acp{PLM} on \acs{NLI}, a zero-shot baseline is tested. As the \ac{PLM} we plan to use (see \autoref{sec:models_datasets} for further information) is a \ac{MLM}, a mask prediction task is used for zero-shot testing. The template used for that task is \texttt{<premise> <mask> <hypothesis>}, where \texttt{<premise>} is the entire premise sentence with the full stop removed, \texttt{<hypothesis>} is the hypothesis sentence with the first word not capitalized and \texttt{<mask>} is the token that will be predicted.

\begin{table}[ht]
    \centering
    \caption{The discourse markers that were chosen a priori for the zero-shot task with their associated labels.}
    \small
    \begin{tabular}{l | l | l}
        \multicolumn{1}{c|}{Entailment} & \multicolumn{1}{c|}{Neutral} & \multicolumn{1}{c}{Contradiction} \\
        \hline
        as & and  & but \\
        because & also & although \\
        so & or & still
    \end{tabular}
    \label{tab:discourse:markers}
\end{table}

This task is inspired by the discourse prediction task introduced by \cite{dissent}, where a word to relate two sentences to each other is to be predicted. In the same way, we constrain the number of words that are relevant for this task to a short list of typical discourse markers shown in \autoref{tab:discourse:markers} and only compare predictions between those. Each discourse marker is associated with a class. The predicted class is then obtained by taking the class of the discourse marker with maximum probability. As the chosen model tokenizes the start of words with an additional \texttt{Ġ}, we add that to all words but do not show it in the table. We ensured that all words are exactly one single token so that simple masked language modeling can be used.

\begin{table}[ht]
    \centering
    \caption{The words chosen by tuning for the zero-shot task with their associated labels.}
    \small
    \begin{tabular}{l | l | l}
        \multicolumn{1}{c|}{Entailment} & \multicolumn{1}{c|}{Neutral} & \multicolumn{1}{c}{Contradiction} \\
        \hline
        Also & Apparently & Yet \\
        Yes & Perhaps & However \\
        More & Clearly & but \\
        Certainly & Obviously & Unfortunately \\
        yes & Presumably & Otherwise \\
        by &  & Except \\
        Specifically &  & no \\
        Indeed &  & Nearly \\
        Yeah &  & Currently \\
        &  & Sadly \\
        &  & Instead \\
        &  & Not \\
        &  & Previously \\
        &  & Until \\
    \end{tabular}
    \label{tab:discourse:markers:tuned}
\end{table}

Moreover, to test possible improvements by choosing other words, two additional methods are tested. First, we test the simplest possible baseline of just the words \enquote{yes} for \texttt{entailment}, \enquote{maybe} for \texttt{neutral} and \enquote{no} for \texttt{contradiction}. This method does not result in grammatical discourses or sentences, but should still provide a viable baseline, as all words result in similarly ungrammatical sentences, but their embeddings differ. The last method is based on tuning the probing, based on a subset of the training set, that is not used during testing. We tune by obtaining, for each label separately, a count of how many times each token occurs in the top three tokens to be predicted with the chosen template. Then we subtract, for each label and each token, the count of the token for the other labels to filter common words. Lastly, we take all sensible words that have a final count of 15 or more, which are shown in \autoref{tab:discourse:markers:tuned}. This way we account for possibly worse performance that is caused by our choice of discourse markers and choose a potentially better set.

\subsection{Fine-tuning for \acs{NLI}}

Fine-tuning for \acs{NLI} is performed by supervised training on a corpus containing premises, hypotheses and the expected labels. To predict on a single sample, both the premise and hypothesis are fed into the network separated by a separator token. The prediction is then computed by a classification head based on the pooled representation of the complete input. Classification is performed by predicting a vector with three dimensions where each dimension corresponds to one of the labels. The predicted label is the index of the maximum value in that vector. Thus the model is fine-tuned by training it to predict the correct label on the training dataset using the cross-entropy between the predicted class vector and the real class vector as the loss function.

\subsection{Detecting biased data} \label{sec:method:detecting_biased_data}

By changing the fine-tuning process to only using the hypothesis as input of the model, biases in the data can be found. Such a hypothesis-only model can only correctly predict the labels either by chance or by abusing biases in the data -- it is never correct for the right reasons. It has been shown that for datasets currently used for fine-tuning for \acs{NLI}, hypothesis-only models can be trained that are better than a majority baseline \citep{hyponly}. Thus, it can be concluded that biases in the data must exist that facilitate correct predictions based only on the hypothesis.

We use this fact to find biased samples in the training datasets. A hypothesis-only model can correctly classify samples based on random chance. This does not indicate bias and must be mitigated. We fine-tune three hypothesis-only models on the dataset with different random seeds. Afterward, we declare all samples biased, that are predicted correctly by at least two of the three hypothesis-only models. We also test only declaring samples biased that are predicted correctly by all three hypothesis-only models. $51.28\%$ are deemed biased by at least two models and only $36.53\%$ by all three models.

\subsection{Mitigating data bias}\label{par:method:mitigating_data_bias}

We employ two methods to remove data bias from the training procedure. The naive method is, to simply remove all samples deemed biased from the training set. By completely removing them from the training procedure, the model cannot be biased by those samples. We call the resulting dataset \texttt{filtered}.

An additional method is introduced by \citet{ensemble}. This method is based on using an ensemble of a frozen biased model and a main model during training and only using the fine-tuned main model during testing. By using the frozen biased model in an ensemble with the main model, the main model can learn to predict based on patterns other than those based on biases. The ensembling can be done by multiplying the prediction of the biased model with the prediction of the main model. The influence of the prediction of the biased model can be reduced by a learned value that is predicted by a secondary head of the main model. By learning to always completely discount the biased model, the model might then learn the biases itself. To prevent this, an additional entropy term is added to the loss function, which punishes the model for discounting the biased prediction too much. As mentioned in the paper, we can also sum the logarithms of the probabilities. We use this, as it should be stabler. \footnote{For more information and justification of this procedure compare with section 3.2 of \cite{ensemble}.}

\subsection{Getting data specific to quantifiers} \label{sec:meth:recasting}

As we will show in \autoref{sec:results}, the model is especially biased for samples on quantifiers. To fight this bias, we create additional training data by recasting question-answering-data to the task of \ac{NLI}. The dataset consists of news articles and single sentences that can be entailed when the placeholder is replaced by a particular entity, but cannot be entailed when it is replaced with a different entity. To ensure that the new examples help with the understanding of quantifiers, we only select those containing quantifiers in both the question and the answer. We detect quantifiers by first checking if the word or words the quantifier is consisting of are contained in the sentence. Then, we check if the words have the correct part of speech tags. We do that, as some words we identify as quantifiers can be used in a different sense, such as \enquote{most}, which can be used as an adjective or as a determiner but is not a quantifier when used as an adjective. We use \texttt{nltk} \cite{nltk} for part of speech tagging.

From each question-answer pair, we try to create an example with gold labels \texttt{entailment}, \texttt{contradiction} and \texttt{neutral} respectively. The premise for each sample is generated by summarizing the question using a fine-tuned \ac{LM}. If the summary returned contains more than one sentence, then one of those needs to be selected, as the premise should always contain exactly one sentence. We select the sentence by first identifying all named entities in the answer and then selecting the summary sentence with the biggest number of entities contained. If no sentence contains a sufficient amount of answer entities, we do not use this sample.

\begin{table}[ht!]
    \centering
    \caption{Corresponding original and replacement quantifier used for generating contradicting sentences.}
    \small
    \begin{tabular}{l | l}
        \multicolumn{1}{c|}{Original} & \multicolumn{1}{c}{Replacement} \\
        \hline
        a &  no \\
        a few &  many \\
        a large number of &  just a small number of \\
        a little &  a lot \\
        a number of &  zero \\
        a small number of &  a large number of \\
        all &  a few \\
        any &  all \\
        both &  neither \\
        every &  none \\
        each &  just a few \\
        enough &  insufficiently many \\
        few &  many \\
        fewer &  more \\
        less &  more \\
        lots of &  no more than a few \\
        most &  least \\
        many &  few \\
        many of &  few of \\
        much &  limited amount of \\
        neither &  both \\
        no &  most \\
        none of &  several \\
        not many &  each \\
        not much &  much \\
        never &  sometimes \\
        numerous &  limited amount of \\
        plenty of &  shortage of \\
        several &  just one \\
        some &  zero \\
        this &  that \\
        that &  this \\
        the &  none of \\
        whole &  only a part
    \end{tabular}
    \label{tab:contradiction:mapping}
\end{table}

The hypothesis is generated differently for each gold label but all have the same premise. The hypothesis for \texttt{entailment} is simply the correct answer provided by the dataset. The contradicting hypothesis is generated from the answer by replacing the quantifier with an opposing quantifier. We selected the opposing quantifiers based on replacement quantifiers not appearing too often and fitting as an antonym for each quantifier. The mapping from the original quantifier to its replacement can be found in \autoref{tab:contradiction:mapping}.

Lastly, sentences with gold label \texttt{neutral} for the given premise are generated from the answer by using quantifier monotonicity properties similar to the approach chosen by \citet{yanaka-etal-2019-help}. Instead of identifying quantifiers by their semantic tags, we identify them using their part of speech tags, as previously described. Afterward, we also generate a \ac{CCG} derivation using the C\&C Tools \cite{curran-etal-2007-linguistically} and try to extract the nominal and verbal phrases that are part of the quantifier phrase. Next, based on the monotonicity we swap the verbs with either hypernyms or hyponyms found using WordNet \cite{miller-1994-wordnet}. If the quantifier is monotonically rising, we use a hypernym to create a situation, where it cannot be said, if the resulting sentence can be entailed by the original sentence. In the same way, for monotonically falling quantifiers, we use hyponyms to create the samples with \texttt{neutral} as the gold label.

\section{Models and Data Sets} \label{sec:models_datasets}
\paragraph{Models}
All experiments and variants are based on the pre-trained \acf{RoBERTa} \citep{roberta} in the variation \texttt{roberta-base}, as this provides a good tradeoff of high downstream performance and lower computational requirements. The default pre-trained model is used for the prompt task. All fine-tuned models are based on the pre-trained model with an additional classification head based on the pooled token representation. The classification head is a multilayer perceptron with a single hidden layer of size $768$ and the pooled representation is obtained from the first \texttt{<s>}-token in the output of the \acs{RoBERTa} model. This \texttt{<s>}-token is the equivalent of \acs{RoBERTa} to the \texttt{CLS}-token of other models.

\begin{table}[h]
    \centering
    \caption{Class distributions for the datasets used}
    \begin{tabular}{r || c | c | c}
        & \acs{MultiNLI} & \acs{SICK} & \acs{e-SNLI} \\
        \hline
        Entailment & $137841$ & $2821$ & $190113$ \\
        Neutral & $137152$ & $5595$ & $189218$ \\
        Contradiction & $137356$ & $1424$ & $189702$
    \end{tabular}
    \label{tab:datasets:classes}
\end{table}

\paragraph{Datasets} We use \acs{MultiNLI} \citep{multinli}, \acs{e-SNLI} \citep{esnli} and \acs{SICK} \citep{sick}. In the following, the datasets are described in more detail. Statistics of the datasets can be seen in \autoref{tab:datasets:classes} and \autoref{tab:datasets:sizes}. \autoref{tab:datasets:classes} gives an overview of the distribution of the classes for the datasets and \autoref{tab:datasets:sizes} an overview of the dataset sizes with their respective dataset splits.

\begin{table}[h]
    \centering
    \caption{Dataset split sizes. \acs{MultiNLI} shows the matched/mismatched validation sizes.}
    \begin{tabular}{r || c | c | c}
        & \acs{MultiNLI} & \acs{SICK} & \acs{e-SNLI} \\
        \hline
        Train & $392702$ & $4439$ & $549367$ \\
        Validation & $9815$/$9832$ & $495$ & $9842$ \\
        Test & - & $4906$ & $9824$
    \end{tabular}
    \label{tab:datasets:sizes}
\end{table}

\Acf{MultiNLI} \citep{multinli} is a very large corpus that improves upon the \acs{SNLI} corpus by collecting premise-hypothesis pairs from ten different domains. Additionally, only five genres are included in the training dataset and two different validation datasets are provided. One of the validation datasets consists of the same genres as the training dataset while the other validation dataset consists of pairs from five different genres. This allows for cross-domain evaluation and comparisons to in-domain evaluation. Furthermore, including training data from multiple genres is hypothesized to reduce linguistic bias.

\begin{lstlisting}[
    language=json,
    caption={Relevant features of a random data sample from \acs{MultiNLI}.},
    label=code:data:samples:multinli
    ]
{
  "hypothesis": "Product and geography are what...",
  "premise": "Conceptually cream skimming has...",
  "label": 1,
  ...
}
\end{lstlisting}

\autoref{code:data:samples:multinli} shows relevant features of a random sample from the \ac{MultiNLI} dataset. Included are the hypothesis and premise as plain text and the expected label numerically encoded. Additional features such as parses of the premise and hypothesis and the genre of the pair are included in the dataset but irrelevant to this project.

\Acf{SICK} \citep{sick} is a small corpus constructed specifically to address issues with crowd-sourced datasets. It is constructed from two source datasets that describe the same videos or images. The descriptions are first normalized and then expanded to include specific linguistic phenomena. The dataset is much smaller than \acs{SNLI} and \acs{MultiNLI} but is considered to have much higher data quality. The features present in \acs{SICK} are similar to the features present in \acs{MultiNLI}, this is the same numerical label, the premise and the hypothesis as plain text. No parses and genre indications are included, but no further interest is spent on this detail, as those features are not relevant to this project.

\Acf{e-SNLI} \citep{esnli} is a variant of the \acs{SNLI} \citep{snli} corpus that adds up to three natural language explanations and for each explanation an annotation of which words in the premise and hypothesis sentences are deemed important for correct classifications.

\begin{lstlisting}[
    language=json,
    caption={A random data sample from \acs{MultiNLI}.},
    label=code:data:samples:esnli
    ]
{
  "explanation_1": "the person is not neces...",
  "explanation_2": "",
  "explanation_3": "",
  "hypothesis": "A person is training his horse...",
  "label": 1,
  "premise": "A person on a horse jumps over a...",
  "sentence1_highlighted_1": "{}",
  "sentence1_highlighted_2": "",
  "sentence1_highlighted_3": "",
  "sentence2_highlighted_1": "3,4,5",
  "sentence2_highlighted_2": "",
  "sentence2_highlighted_3": ""
}
\end{lstlisting}

\autoref{code:data:samples:multinli} depicts a random sample from the \acs{e-SNLI} corpus including all available features. Comparing it to the features of \acs{MultiNLI} and \acs{SICK}, it is obvious that explanations and highlights of the sentences are added to the data. Up to three different explanations are included and for each explanation, the words in the premise and hypothesis that are relevant to this explanation are provided. The words are provided as indices into the premise and hypotheses starting at zero.

\section{Experiments}

\paragraph{Baseline} We use two models as baselines to compare against our results. A pre-trained \ac{RoBERTa} model will serve as a zero-shot baseline. Furthermore, we use a \ac{RoBERTa} model fine-tuned on the \ac{MultiNLI} dataset as a fine-tuned baseline. The comparison between our fine-tuned results and the zero-shot baseline is used to test \textbf{H1}.

\paragraph{Obtaining models with lower bias} To pursue our two approaches to obtain a fine-tuned model with less bias we conduct the following experiments: For the first approach, we fine-tune a pre-trained \ac{RoBERTa} model on the \ac{MultiNLI} dataset after it has been preprocessed as described in \autoref{sec:method} to reduce the bias in the dataset. Optionally, we can add the training data from \ac{SICK} to increase our total amount of training data.

For the second approach we need to conduct two experiments: First, we fine-tune a pre-trained \ac{RoBERTa} model only on the hypotheses of the entire \ac{MultiNLI} dataset to create a biased model. Then an ensemble consisting of the biased model we previously obtained and a standard \ac{RoBERTa} model is trained on the entire \ac{MultiNLI} dataset as described in \autoref{sec:method}.

The comparisons between both approaches to the fine-tuned baseline are used to test \textbf{H2}.

\paragraph{Test} To measure the quality of the models, they are tested in two different aspects: First, the predictive performance that the models achieve is measured by testing them on the SICK dataset as it is less biased than the \ac{e-SNLI} dataset. Additionally, it is ensured that the models come to their predictions for the right reasons by applying the interpretability methods described in \autoref{sec:analysis}. These tests are conducted on subsets of the \ac{e-SNLI} dataset. Each subset contains only records that represent a particular linguistic phenomenon. This partitioning of the dataset allows for determining which linguistic phenomena the respective model copes better with or worse.

The comparison between the models and the fine-tuned baseline on the \ac{e-SNLI} dataset is used to test \textbf{H3}. To test \textbf{H4}, we compare the models to the fine-tuned baseline on the \ac{SICK} dataset.
\section{Results and Analysis} \label{sec:results}
% 1. Accuracy (F1 + MCC) auf SICK und eSNLI einzeln nach Kategorien
% 2. Auf Bias überprüfen: Vergleich vom Modell für wichtig erachtete Token mit von Menschen als wichitg erachtete Tokens
% Visualisierungen:
% - Confusion Matrix (Gentrennt nach Phänomenen)
% - Tabellen Interpretability Metriken (siehe ferret)

\paragraph{Detecting biases in data}
Figure~\ref{fig:metric-heatmap-phenomena-mcc} depicts the Matthews Correlation Coefficients for our model trained on different datasets separated by linguistic phenomena: \enquote{default} is the model trained on the regular \ac{MultiNLI} dataset, \enquote{filtered} trained on samples which where correctly classified of maximum two hypothesis only models and \enquote{hypothesis-only} is the model trained only on the hypotheses of \ac{MultiNLI}.

\begin{figure}[ht]
    \centering
    \includegraphics[width=0.9\columnwidth]{./images/metric_heatmaps_phenomena/important_words/matthews_correlation.pdf}
    \caption{Matthews Correlation Coefficients for the model trained on different datasets separated by linguistic phenomena}
    \label{fig:metric-heatmap-phenomena-mcc}
\end{figure}

The hypothesis-only model performs poorly as expected but also comparably well on samples that contain antonyms which indicates a bias in these samples and thus proves \textbf{H2}. The improved performance of the filtered model also shows that this bias can be mitigated by filtering the biased samples and thus proves \textbf{H4}.

The hypothesis-only model also performs better on samples that contain quantifiers which indicates a bias also in these samples. The filtered model performs slightly worse on quantifiers than the default model. Thus, it is not sufficient to filter out the biased samples to increase performance on quantifiers. A reason for this behavior might be the greater complexity of quantifiers compared to antonyms. To increase the model's performance on quantifiers nevertheless, we collect additional training samples containing quantifiers.

\begin{figure*}[h!]
    \centering
    \includegraphics[width=\textwidth]{./images/ferret_sample.pdf}
    \caption{Expamle of a bad explanation}
    \label{fig:ferret-sample}
\end{figure*}

\FloatBarrier{}

\paragraph{Detecting biases in models}
Figure~\ref{fig:ferret-sample} shows the explanations for the default model's classification of a sample from the validation split of the \ac{e-SNLI} dataset obtained using ferret \cite{ferret}. It can be seen that the meaning of the quantifiers in this sample is not well captured by the model: The usage of quantifiers in this sample makes it a contradiction but the model only attaches comparably high importance to the quantifier \enquote{all} whereas \enquote{several} only has very low importance to the model. The existence of such samples where the model only poorly captures the importance of quantifiers proves \textbf{H2}.

To demonstrate the biases, we provide example sentences with important tokens highlighted. Furthermore, to provide a different view on the importance of certain tokens, we visualize the attention maps.

\begin{table}[ht!]
    \centering
    \caption{Prediction performance when using prompting with different word groups on the \acs{SICK} dataset. The best result is shown in \textbf{bold} and the second-best is \underline{underlined}.}
    \begin{tabular}{l c c}
        \toprule
        \multicolumn{1}{c}{Word Group} & \acs{MCC} & $\text{F}_1$ \\
        \midrule
        A Priori & $21.54\%$ & $\mathbf{52.87\%}$ \\
        Simple & $\underline{22.71\%}$ & $\underline{36.99\%}$ \\
        Tuned & $\mathbf{30.34\%}$ & $33.87\%$ \\
        \bottomrule
    \end{tabular}
\end{table}

\begin{table}[ht!]
    \centering
    \caption{Prediction performance of our fine-tuned models on the \acs{SICK} dataset. The best result is shown in \textbf{bold} and the second-best is \underline{underlined}.}
    \begin{tabular}{l c c}
        \toprule
        \multicolumn{1}{c}{Model} & \acs{MCC} & $\text{F}_1$ \\
        \midrule
        Base & $49.49\%$ & $56.60\%$ \\
        Hypothesis-Only\tablefootnote{Average of three runs with different seeds} & $14.13\%$ & $40.02\%$ \\
        Filtered $2/3$ & $46.21\%$ & $54.12\%$ \\
        Filtered $2/3$ longer & $36.17\%$ & $52.85\%$ \\
        Filtered $3/3$ & $48.73\%$ & $56.94\%$ \\
        Filtered $3/3$ longer & $\mathbf{52.31\%}$ & $\mathbf{62.04\%}$ \\
        Ensembled & $\underline{51.65\%}$ & $\underline{59.88\%}$ \\
        Recast & n/a & n/a \\
        \bottomrule
    \end{tabular}
\end{table}

\section{Conclusion} \label{sec:conclusion}
wefw

\FloatBarrier
\bibliography{lib}
\bibliographystyle{acl_natbib}

\FloatBarrier
\appendix
\section{Bias plots for the base finetuned model} \label{sec:bias_plots_base}

The faithfulness and plausibility metrics of the base fine-tuned model obtained using ferret are depicted in \autoref{fig:ferret-base}. The explainers are listed on the x-axis. \enquote{Ig} is shorthand for the integrated gradient explainer. \enquote{Igmby} is the integrated gradient explainer multiplied by the inputs.

\begin{figure*}[t!]
    \centering
    \begin{subfigure}{0.49\textwidth}
        \includegraphics[width=\textwidth]{./images/ferret_heatmaps_phenomena/default_mnli/synonym.pdf}
        \caption{Synonyms}
    \end{subfigure}
    \begin{subfigure}{0.49\textwidth}
        \includegraphics[width=\textwidth]{./images/ferret_heatmaps_phenomena/default_mnli/antonym.pdf}
        \caption{Antonyms}
    \end{subfigure}
    \begin{subfigure}{0.49\textwidth}
        \includegraphics[width=\textwidth]{./images/ferret_heatmaps_phenomena/default_mnli/hypernym.pdf}
        \caption{Hypernyms}
    \end{subfigure}
    \begin{subfigure}{0.49\textwidth}
        \includegraphics[width=\textwidth]{./images/ferret_heatmaps_phenomena/default_mnli/hyponym.pdf}
        \caption{Hyponyms}
    \end{subfigure}
    \begin{subfigure}{0.49\textwidth}
        \includegraphics[width=\textwidth]{./images/ferret_heatmaps_phenomena/default_mnli/co_hyponym.pdf}
        \caption{Co-Hyponyms}
    \end{subfigure}
    \begin{subfigure}{0.49\textwidth}
        \includegraphics[width=\textwidth]{./images/ferret_heatmaps_phenomena/default_mnli/quantifiers.pdf}
        \caption{Quantifiers}
    \end{subfigure}
    \begin{subfigure}{0.49\textwidth}
        \includegraphics[width=\textwidth]{./images/ferret_heatmaps_phenomena/default_mnli/numericals.pdf}
        \caption{Numerals}
    \end{subfigure}
    \caption{Faithfulness and plausibility of the base model measured on \acs{e-SNLI} filtered for linguistic phenomena}
    \label{fig:ferret-base}
\end{figure*}

\section{Bias plots for the filtered finetuned model} \label{sec:bias_plots_filtered}

The faithfulness and plausibility metrics of the model \enquote{Filtered 3/3 longer} obtained using ferret are depicted in \autoref{fig:ferret-base}. The explainers are listed on the x-axis. \enquote{Ig} is shorthand for the integrated gradient explainer. \enquote{Igmby} is the integrated gradient explainer multiplied by the inputs.

\begin{figure*}[t!]
    \centering
    \begin{subfigure}{0.49\textwidth}
        \includegraphics[width=\textwidth]{./images/ferret_heatmaps_phenomena/filtered_3_3_longer/synonym.pdf}
        \caption{Synonyms}
    \end{subfigure}
    \begin{subfigure}{0.49\textwidth}
        \includegraphics[width=\textwidth]{./images/ferret_heatmaps_phenomena/filtered_3_3_longer/antonym.pdf}
        \caption{Antonyms}
    \end{subfigure}
    \begin{subfigure}{0.49\textwidth}
        \includegraphics[width=\textwidth]{./images/ferret_heatmaps_phenomena/filtered_3_3_longer/hypernym.pdf}
        \caption{Hypernyms}
    \end{subfigure}
    \begin{subfigure}{0.49\textwidth}
        \includegraphics[width=\textwidth]{./images/ferret_heatmaps_phenomena/filtered_3_3_longer/hyponym.pdf}
        \caption{Hyponyms}
    \end{subfigure}
    \begin{subfigure}{0.49\textwidth}
        \includegraphics[width=\textwidth]{./images/ferret_heatmaps_phenomena/filtered_3_3_longer/co_hyponym.pdf}
        \caption{Co-Hyponyms}
    \end{subfigure}
    \begin{subfigure}{0.49\textwidth}
        \includegraphics[width=\textwidth]{./images/ferret_heatmaps_phenomena/filtered_3_3_longer/quantifiers.pdf}
        \caption{Quantifiers}
    \end{subfigure}
    \begin{subfigure}{0.49\textwidth}
        \includegraphics[width=\textwidth]{./images/ferret_heatmaps_phenomena/filtered_3_3_longer/numericals.pdf}
        \caption{Numerals}
    \end{subfigure}
    \caption{Faithfulness and plausibility of the \enquote{Filtered 3/3 longer} model measured on \acs{e-SNLI} filtered for linguistic phenomena}
    \label{fig:ferret-filtered}
\end{figure*}

\end{document}
