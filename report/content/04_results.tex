\section{Results and Analysis} \label{sec:analysis}
% 1. Accuracy (F1 + MCC) auf SICK und eSNLI einzeln nach Kategorien
% 2. Auf Bias überprüfen: Vergleich vom Modell für wichtig erachtete Token mit von Menschen als wichitg erachtete Tokens
% Visualisierungen:
% - Confusion Matrix (Gentrennt nach Phänomenen)
% - Tabellen Interpretability Metriken (siehe ferret)


To analyze the performance of the \acp{LM}, we test their predictive performance and their bias. We provide different scores and visualizations to test our hypothesis and provide insights into the datasets and \acp{LM}.
% Accuracy

To measure the predictive performance, we provide the macro $\text{F}_1$-score \citep{macrof1} and \ac{MCC} \citep{mcc}, as \ac{MCC} has proven to be more reliable than accuracy and $\text{F}_1$ \citep{mccGood}.

% Bias
To analyze biases, we compare the words deemed important by the \ac{e-SNLI} explanations to the tokens deemed important for the model prediction by interpretability methods. Furthermore, to detect linguistic biases, we conduct a confusion analysis for each specific linguistic phenomenon.

% Ferret
To generate explanations we use the interpretability methods Integrated Gradients \citep{integratedgradients}, \ac{LIME} \citep{lime} and Partition SHAP Values \citep{shap}. We use the plausibility and faithfulness \citep{ferret} metrics to judge the model bias.

To demonstrate the biases, we provide example sentences with important tokens highlighted. Furthermore, to provide a different view on the importance of certain tokens, we visualize the attention maps.