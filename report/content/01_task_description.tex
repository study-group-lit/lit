\section{Motivation}

\ac{NLI} is the task of deciding the truthiness of a hypothesis, given a premise. If the hypothesis is true, it can be said to be \textbf{entailed} by the premise. If it is false, it is \textbf{contradictory} to the premise. Else the truthiness of the hypothesis cannot be determined. For each case, respectively one of the three classes \texttt{entailment}, \texttt{contradiction} or \texttt{neutral} is chosen.

% What is our aim?
Our aim is to identify biases in \acp{LM} for \ac{NLI} and mitigate those biases by removing biased data in the training procedure.

% What are our hypotheses?
We pose the following hypotheses:\vspace{-1.5em}
\begin{description}
  \item[H1] The fine-tuning procedure of \acp{LM} introduces biases for the \ac{NLI} task.\vspace{-0.7em}
  \item[H2] Mitigating biases from the datasets results in less biased \acp{LM}.\vspace{-0.7em}
  \item[H3] \acp{LM} with less bias make worse predictions for biased data. \vspace{-0.7em}% Will not name accuracy here, as accuracy is bad. We can name F1 and other score later.
  \item[H4] \acp{LM} with less bias make better predictions for data without the same bias.
\end{description}

w\cite{dissent}