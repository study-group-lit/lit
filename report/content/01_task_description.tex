\section{Introduction}


\ac{NLI} is the task of deciding the truthiness of a hypothesis, given a premise. If the hypothesis is true, it can be said to be entailed by the premise. If it is false, it is contradictory to the premise. Otherwise, the truthiness of the hypothesis cannot be determined. For each case, respectively one of the three classes \texttt{entailment}, \texttt{contradiction} or \texttt{neutral} is chosen.

\subsection{Motivation}
The ultimate goal of research in natural language semantics is to represent the meaning of language and how conclusions can be drawn from it. From this point of view, the task of deciding if a sentence in natural language can be entailed by another sentence in natural language is useful and methods to perform such a task could be used as building blocks for many other natural language processing tasks. For example, it has been shown, that a model pre-trained on \ac{NLI} can be used as an effective zero-shot text classifier \cite{yin-etal-2019-benchmarking}.

For \ac{NLI}, complex reasoning is needed in many cases, as minute differences can change the label from \texttt{entailment} to \texttt{neutral} or vice-versa. Furthermore, it has been shown multiple times, that current approaches to \ac{NLI} have many biases in models and datasets and can be easily fooled \cite{hyponly,gururangan-etal-2018-annotation,glockner-etal-2018-breaking}.

Considering the importance of \ac{NLI} for natural language understanding and the current limitations, we set out to improve the state of the art.

\subsection{Research Goals}
As \ac{NLI} is an important task to verify the reasoning capability of \acp{LM}, we want to further analyze it. To this end, we want to show limitations in the current approach to \ac{NLI} and then investigate mitigations to obtain better models.

Thus, our goal is to identify linguistic biases in \acp{LM} for \ac{NLI}, then mitigate those biases by removing biased data in the training procedure and recasting data from other domains to combat the specific biases.
\vspace{1em}

\noindent
We pose the following hypotheses: \vspace*{-0.5em}
\begin{description}
  \setlength\itemsep{-0.3em}
  \item[H1] \acp{PLM} do not contain enough inherent information without fine-tuning for \ac{NLI}.
  \item[H2] Fine-tuned models are biased for some linguistic phenomena.
  \item[H3] The chosen training data is biased and the found bias is not distributed uniformly over linguistic phenomena.
  \item[H4] Mitigating biased data during training results in models with greater predictive performance and less bias.
  \item[H5] Using additional data during training for a specific linguistic phenomenon the model is biased for results in models with greater predictive performance and less bias.
\end{description}

\subsection{Structure of our report}
In the rest of this report, we describe our methods in \autoref{sec:method} and thoroughly describe the used models and datasets in \autoref{sec:models_datasets}. We present our experimental setup in \autoref{sec:experiments} and analyze the results in \autoref{sec:results}. Lastly, we formulate our findings regarding the posed hypotheses in \autoref{sec:conclusion}.